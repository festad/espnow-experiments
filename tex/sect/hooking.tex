Here we talk about hooking.


\newsavebox\funlmactxframe

\lstset{
	basicstyle=\ttfamily,
	columns=fullflexible,
	%frame=single,
	breaklines=true,
	postbreak=\mbox{\textcolor{red}{$\hookrightarrow$}\space},
}

\begin{lrbox}{\funlmactxframe}
\begin{lstlisting}
void lmacTxFrame(int param_1,int param_2)
{
    /*Decompiled code from the original libraries*/
    4000a06a 93 97 49 03 li     a5,0x34
    4000a06e b3 87 f5 02 mul    a5,a1,a5
    4000a072 01 11       c.addi sp,-0x20
    4000a074 22 cc       c.swsp s0,0x18(sp)
    4000a076 37 04 88 40 lui    s0,0x40880
    4000a07a 52 c4       c.swsp s4,0x8(sp)
    ...
}
\end{lstlisting}
\end{lrbox}

\newsavebox\funpatchlmactxframe
\begin{lrbox}{\funpatchlmactxframe}
\begin{lstlisting}
void patched_lmacTxFrame(int param_1,int param_2)
{
    /*Our patched version of lmacTxFrame*/
}
\end{lstlisting}
\end{lrbox}

\newsavebox\funactivation
\begin{lrbox}{\funactivation}
\begin{lstlisting}
void activation()
{
    /*Our activation function*/
    // lui+jalr to call_lmacTxFrame
    uint32_t lui_instr  = 0x4200c0b7;   // LUI instruction
    uint32_t jalr_instr = 0x1b4080e7;  // JALR instruction

    uint32_t *base_ppProcessTxQ = (uint32_t *) 0x4080e3fc;
    uint32_t offset_ppProcessTxQ = 0x70; // 0x4080e46c - 0x4080e3fc

    // ppProcessTxQ
    uint32_t *ppProcessTxQ_lui = \
        (uint32_t *)((char *)base_ppProcessTxQ +\
                     offset_ppProcessTxQ);
    uint32_t *ppProcessTxQ_jalr = \
        (uint32_t *)((char *)base_ppProcessTxQ +\
                     offset_ppProcessTxQ + 0x4);
    *ppProcessTxQ_lui = lui_instr;
    *ppProcessTxQ_jalr = jalr_instr;    
}
\end{lstlisting}
\end{lrbox}

\newsavebox\funcalllmactxframe
\begin{lrbox}{\funcalllmactxframe}
\begin{lstlisting}
/*Suppose the compiled code starts at 
  0x4200c1b4*/
void call_lmacTxFrame(int param_1,int param_2)
{
    /*Our function to call lmacTxFrame*/
    counter++; // A global counter of how 
               // many times lmacTxFrame is called
    /*Do some processing such as changing the
      channel before calling lmacTxFrame*/
    return lmacTxFrame(param_1,param_2);
    /* Or return patched_lmacTxFrame(param_1,param_2)
       in case we want to patch the original function*/
}
\end{lstlisting}
\end{lrbox}

\newsavebox\funprocesstxq
\begin{lrbox}{\funprocesstxq}
\begin{lstlisting}
void processTxQ()
{
    /*One among different functions from the 
      original library that calls lmacTxFrame*/
    4080e3fc 41 11       c.addi  sp,-0x10
    4080e3fe 26 c2       c.swsp  s1,0x4(sp)
    4080e400 06 c6       c.swsp  ra,0xc(sp)
    4080e402 22 c4       c.swsp  s0,0x8(sp)
    4080e404 aa 84       c.mv    s1,a0
    ... 
    4080e46c b7 a0 00 40 lui     ra,0x4000a
    4080e470 e7 80 a0 06 jalr    ra,ra,offset=0x06a lmacTxFrame 
    ...
}
\end{lstlisting}
\end{lrbox}


\newsavebox\funprocesstxqaft
\begin{lrbox}{\funprocesstxqaft}
\begin{lstlisting}
void processTxQ()
{
    /*One among different functions from the 
      original library that calls lmacTxFrame*/
    4080e3fc 41 11       c.addi  sp,-0x10
    4080e3fe 26 c2       c.swsp  s1,0x4(sp)
    4080e400 06 c6       c.swsp  ra,0xc(sp)
    4080e402 22 c4       c.swsp  s0,0x8(sp)
    4080e404 aa 84       c.mv    s1,a0
    ... 
    4080e46c b7 c0 00 42 lui     ra,0x4200c
    4080e470 e7 80 40 1b jalr    ra,ra,offset=0x1b4 call_lmacTxFrame 
    ...
}
\end{lstlisting}
\end{lrbox}

\newsavebox\funpropcode
\begin{lrbox}{\funpropcode}
\begin{lstlisting}
{
/* Proprietary code that calls 
   different functions among which
   those that call lmacTxFrame,
   such as ppProcessTxQ */
     ...
     ppProcessTxQ();
     ...
}
\end{lstlisting}
\end{lrbox}


\newsavebox\funmain
\begin{lrbox}{\funmain}
\begin{lstlisting}
void main()
{
    ...
    /* Before activation, the call flow
       is ppProcessTxQ -> lmacTxFrame */
    activation();
    /* After activation, the call flow
       is ppProcessTxQ -> call_lmacTxFrame
          -> lmacTxFrame (or patched_lmacTxFrame) */
    ...
}
\end{lstlisting}
\end{lrbox}


Here we show the use of an activation function to 
modify the flow of execution of a program by modifying 
directly original assembly code.

\begin{figure*}[ht]
\begin{tikzpicture}[scale=0.55, 
		    every node/.append style={transform shape}]
	\node[draw] (nfunmain) {\usebox\funmain};
	\node[draw, below=5cm of nfunmain] (nfunactivation) {\usebox\funactivation};
	\node[draw, right=6cm of nfunmain] (nfunprocesstxq) {\usebox\funprocesstxq};
	\node[draw, above=of nfunprocesstxq] (nfunpropcode) {\usebox\funpropcode};
	\node[draw, below=of nfunprocesstxq] (nfunlmactxframe) {\usebox\funlmactxframe};
	\node[draw, below=of nfunlmactxframe] (nfunprocesstxqaft) {\usebox\funprocesstxqaft};
	\node[draw, below=of nfunprocesstxqaft] (nfuncalllmactxframe) {\usebox\funcalllmactxframe};
	%\node[draw, below left=of nfuncalllmactxframe] (nfunlmactxframe) {\usebox\funlmactxframe};
	\node[draw, below=of nfuncalllmactxframe] (nfunpatchlmactxframe) {\usebox\funpatchlmactxframe};

	% using to for the edge between the nodes with in and out angles and text auto sloped
	\draw[->] (nfunmain) to[out=-100,in=100] node[auto,sloped] {} (nfunactivation);
	\draw[->, color=red] (nfunactivation) to[out=90,in=200] node[auto,sloped] {Injecting assembly} (nfunprocesstxq);
	\draw[->, color=red] (nfunprocesstxq) to[out=20, in=20, looseness=0.8] node[auto,sloped] {} (nfunprocesstxqaft);

	% Before activation
	\draw[->] (nfunpropcode) to[out=0, in=10] node[auto,sloped] {before \texttt{activation()}} (nfunprocesstxq);
	\draw[->] (nfunprocesstxq) to[out=0, in=10] node[auto,sloped] {} (nfunlmactxframe);
	\draw[->, color=blue] (nfunpropcode) to[out=10, in=10,looseness=0.8] node[auto,sloped,pos=0.2] {After \texttt{activation()}} (nfunprocesstxqaft);
	\draw[->, color=blue] (nfunprocesstxqaft) to[out=0, in=10] node[auto,sloped,] {} (nfuncalllmactxframe);
	\draw[->, color=blue] (nfuncalllmactxframe) to[out=180, in=180,looseness=0.5] node[auto,sloped] {} (nfunlmactxframe);
	

\end{tikzpicture}

\caption{Original function processTxQ and the function after activation}
\label{fig:workflowchange}

\end{figure*}

