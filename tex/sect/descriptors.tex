The libraries developed by Espressif create a linked structure of descriptors that
is used to point to the content of the bytes being sent or received.
Every packet is described by a triplet: a description of the packet, a pointer to the start of the packet
and a pointer to the next descriptor. Each element of the triplet is a 32-bit word, unfortunately
at the moment of writing it is not completely clear how the values in the description field are used
but it is enough for us to be able to send a custom frame with a length of our choice.
\newsavebox{\descriptor}
\begin{lrbox}{\descriptor}
\begin{lstlisting}
typedef struct __attribute__((packed)) descriptor {
	uint16_t _unknown_2 : 14;
	uint16_t length : 12;
	uint8_t _unknown_1 : 4;
	uint8_t has_data : 1;
	uint8_t owner : 1;
	void* packet;
	struct descriptor* next;
} descriptor;
\end{lstlisting}
\end{lrbox}
\usebox{\descriptor}
The first word of the descriptor, that is the description, is a 32-bit field 
that contains a bit indicating whether the packet is directed to the device,
it is relevant during the reception of frames and, unless we
are requiring a \textit{promiscuous mode}, when set to 0 the frame is discarded.
The second bit is used to indicate whether the bytes in the area actually contain 
meaningful data or if the content is garbage. The meaning of the following 4 bits, at the moment
of writing, is not clear. 
The length field is a 12-bit field that indicates the length of the frame and the last 14 bits,
at the moment of writing, are not clear.
The length represented in the 12 bits is comprehensive of the radiotap header, 
the 802.11 radio information, the frame content and the FCS.
In actuality, the lenght field in the description must be set to the length of the final desired
length minus 7, that is, when the description field is \texttt{0xfff} (4095) the length of the frame
will be 4102 bytes. The reason for this discrepancy of 7 bytes is not clear at the moment of writing.
It is important to notice how this is only one of the different places where the length of the frame
will be indicated, another place is in the first 4 bytes of the area referenced by the pointer to the packet:
that area starts with 2 words and the last 12 bits of the first word (little-endian speaking)
represent the length of the payload of the frame, that is, the length of the whole frame minus 15 bytes. 
Interestingly enough the radiotap header is not always 15 bytes long, but even when a 18 bytes 
long radiotap header is observed, the length of the payload is still 15 bytes less than the length of the frame.
A 18 bytes long radiotap header is observed when the data rate is set to higher values such as 54 Mbps
and modulation techniques such as OFDM are used instead of DSSS as in the case of data rates such as 1 or 2 Mbps. 
% VERIFY THIS
% VERIFY THIS
The third place where the length of the frame is indicated is in the area of memory reserved to
peripheral registers, the address \texttt{0x600a5488} contains, in its last 12 bits (little-endian speaking),
the length of the payload of the frame (the length of the frame minus 15 bytes, the same value observed
at the beginning of the packet area).
